I conducted some tests in order to explore the algorithm behind the Twinwords Text Similarity Matching API and to get a deeper understanding of the return value.

\textbf{Test case}: Same texts

Text 1: Once upon a time in Holywood

Text 2: Once upon a time in Holywood

Result: 1

Observation: Identical texts have a relationship score of 1.

\textbf{Test case}: Synonyms

Text 1: angry

Text 2: enraged

Result: 0.77

Observation: Synonyms can be recognized.

\textbf{Test case}: Antonyms

Text 1: good

Text 2: bad

Result: 0.22

Observation: Antonyms have lower score.

\textbf{Test case}: Same topic, different categories

Text 1: PC

Text 2: Mac

Result: 0

Observation: Categories are not always picked up correctly.

\textbf{Test case}: Same words, different language

Text 1: almás tészta

Text 2: apple cobbler

Result: 0

Observation: It is not able to translate words. It also does not give a negative score in case the words are in a different language.

\textbf{Test case}: Two words in Hungarian

Text 1: tészta

Text 2: pite

Result: 0

Observation: It can not compare words in Hungarian.

\textbf{Test case}: Two similar texts in hungarian

Text 1: almás pite

Text 2: almás tészta

Result: 0.5

Observation: In this case it compares the two terms in a lexical manner. The first two words are identical, half of the text matches, that is why the score is 0.5

\textbf{Test case}: Two BPMN Processes.

Text 1: (Application processing) File the correspondence. Application received. Post the policy documents. Non-exclusive Gateway. Non-exclusive Gateway. End Event. Issue the policy. Print the address. 

Text 2: (Loan increase application) End Event. process application. Check customer file. Contact the customer. Is the application complete?. Is the signature valid?. Examine the application. Check the signature. receive application. yes. no. no. yes. 

Result: 0.62

Observation: These two processes are somewhat similar, as both are about application processing. They have identical words (file, process, application, receive) and these were identified by the algorithm.

\textbf{Test case}: Two BPMN Processes.

Text 1: (Application processing) File the correspondence. Application received. Post the policy documents. Non-exclusive Gateway. Non-exclusive Gateway. End Event. Issue the policy. Print the address. 

Text 2: (Car re-allocation) 'Reason?. Is a new car needed?. Examine the reason for reallocation. End Event. Cancel because of company's decision. Cancel due to the contract being up. Cancel due to accident. Allocate a new car. Receive car reallocation
claim. contract is up. no. decision by the company. yes. car was in an accident. 

Result: 0.38

Observation: These two processes are not that similar, they do not contain that many identical words. And there are few words which belong to the same category.


\subsection{Conclusion}

The algorithm returns values between 0 and 1 where 1 means that the texts are identical and where 0 means that the texts are completely different. A score of 0 is produced when the input texts are in different languages or both of them are not in English. The algorithm can recognize synonyms and words that belong to the same topic.

The algorithm is invariant to the words' order.